\documentclass[12pt, a4paper]{article}
%
% DECC colour scheme, layout, and other niceties
\usepackage{DECC}
%
% Use Latin Roman fonts and unicode input
\usepackage[utf8]{inputenc}          
\usepackage[T1]{fontenc}              
\usepackage[tt={monowidth}]{cfr-lm} 
%
% Personal preferences
\usepackage{amsmath}
\usepackage{booktabs}
\usepackage{graphicx}
\usepackage[font = small, margin = 8.75mm]{caption}
\usepackage[load-configurations=abbreviations, per-mode=symbol, mode=text]{siunitx}
\DeclareSIUnit{\Wh}{Wh}
\DeclareSIUnit{\kWh}{\kilo\Wh}\DeclareSIUnit{\MWh}{\mega\Wh}\DeclareSIUnit{\GWh}{\giga\Wh}
\DeclareSIUnit{\year}{y}
\usepackage{microtype} % Because we care
%
%
\title{The Value of Storage}
\author{James Geddes} 
\protectivemarking{Unclassified}
\begin{document}
\maketitle

\section{Introduction}
Suppose that a large amount of pumped storage—perhaps equal to another
Dinorwig—were added to the electricity system for free. Whatno
 is the
value of this storage? That is, what is the net benefit to society of
the addition, if it could be done with no cost? This note is an
attempt to see how far we can get in answering that question without
resorting to a detailed model.


\section{Overview of the approach}

If there is value from adding storage, where does it come from? In the
\emph{short-run} (that is, without changing the existing
infrastructure) the use of storage may replace high--variable cost
generation with low--variable cost generation. In the \emph{long-run}
(that is, allowing the optimum configuration of the system to adjust
to the presence of storage) there will be additional value as a
cheaper mix of generation will be available.

Now, the value of storage per ``unit'' of storage, will change if a
sufficiently large amount of storage is added. Throughout this note,
we will assume instead that the amount of storage added is ``small'',
in whatever sense is appropriate.\footnote{What ``small'' means is a
  little unclear. For calculations of the short-run, it presumably
  means that prices are not affected.} More generally, the value of
storage will depend upon the initial state of the system.

Specifically, in the short-run, there are three sources of value:
\begin{enumerate}
\item The transfer of energy from times when low-cost generation is
  under-used to times when high-cost generation is otherwise required;
\item The provision of some kinds of reserve, particularly frequency
  response;
\item Potentially, removal of some local transmission constraints on
  the boundary of the region containing the storage. 
\end{enumerate}

In the long-run, the optimum configuration of the system will
change---presumably, there will be more baseload plant and less
peaking plant---and the whole system will be lower cost that it would
have been otherwise. 

In this note, we consider only the arbitrage value: that is, we do not
consider value from the provision of reserve or removal of
transmission constraints.

\subsection{The parameters of storage}

The ``size'' of storage is characterised by two numbers: its ``power
capacity,'' $C_P$, the rate at which energy can be pumped in to or out
of storage, and its ``storage capacity,'' $C_E$, the total amount of
energy that can be stored. (This is in contrast to generation, to
which only the power capacity applies.) Dinorwig, for example, has
$C_P \approx \SI{1.8}{\GW}$ and $C_E\approx \SI{9.1}{\GWh}$.

In general, the optimum use of a particular storage device depends in
some complicated way on these two parameters. We shall consider two
simplified kinds: one which is \emph{rate limited} (that is, ``takes a
long time to charge up'') and one which is \emph{volume limited}
(that is, ``becomes full very quickly''). Here, ``quickly'' or ``a
long time'' is with reference to the timescales over which prices tend
to change. At present, demand (and hence price) typically varies
strongly over the course of a day (though it also changes over the
year) and we shall use a reference period of a day. Thus, a
rate-limited storage unit is one for which $C_E/C_P \gg \SI{1}{\day}$
and a volume-limited unit is one for which $C_E/C_P \ll
\SI{1}{\day}$. Dinorwig is somewhat volume-limited under this
definition, having $C_E/C_P \approx \SI{5}{\hour}$.

Storage is also affected by the loss of energy during the round-trip
from taking in energy to producing it again. A typical storage unit is
about~\SI{75}{\percent} efficient, which is the figure we will use in
this note.


\section{The short run}

\subsection{Rate-limited storage}

One immediate value of storage comes from moving some energy from
times when it is cheap to generate to times when it is costly to
generate. One way to get a handle on this is to look at prices.
 
If we pretend that the demand curve is completely inelastic then we
might assume that, except when there is shortage, the spot price of
electricity is equal to the short-run marginal cost of
generation. Under this assumption, the revenue accruing to a storage
unit from buying low and selling high is equal to the reduction in
generation cost.
 
\begin{figure}[htb]
\centering
\includegraphics{pd.pdf}
\caption{Price-duration curve for NETA from 2010. Shown also are
  approximate marginal costs of generation (including fuel costs and
  carbon costs, but excluding start-up costs) based on average fuel
  costs over the year. Source: Price shown is the Market Index price
  from Elexon (based on intra-day trades); fuel costs from DECC
  Quarterly Energy Prices; emissions factors from Defra; power station
  efficiencies from DUKES.}
\label{fig:pd}
\end{figure}

Figure~\ref{fig:pd} shows the price-duration curve from 2010. Most of
the time, the price is set by the marginal cost of generation (the
interquartile range over the whole year is \SI{34}[\pounds]{\per\MWh}
to \SI{46}[\pounds]{\per\MWh}) with very occasional periods of much
higher prices (the maximum over the year was
\SI{326}[\pounds]{\per\MWh}. We can estimate how much revenue a
storage unit could earn, by assuming that it buys low and sells
high. If the storage is rate limited it could, in principle, buy as
much as possible at the lowest price of the year and sell at the
highest; buy at the second-lowest price and sell at the
second-highest; and so on, at least until the round-trip losses made
the sale unprofitable.

For 2010, a storage unit following this procedure would have earned,
in revenue, \SI{27200}[\pounds]{\per\MW}, or
\SI{27.2e6}[\pounds]{\per\GW}. (The value is given per gigawatt,
because we are assuming the storage unit is rate limited and therefore
stores or discharges \SI{1}{\GWh} during each hour of operation for
each \si{\GW} of power capacity.) By way of comparison, the amortised
cost, over~20 years, of a \SI{1e9}[\pounds]{} capital investment at a
\SI{10}{\percent} discount rate is about \SI{110e6}[\pounds]{} per
year. At \SI{3.5}{\percent} over 40 years (Dinorwig is already 30
years old) the amortised cost is \SI{46e6}[\pounds]{} per
year. Presumably this is why there's not an awful lot of pumped
storage around at present.

We can repeat the calculation for other years. At present, we have
price data for complete years from 2004 to
2011. Table~\ref{tab:values} shows the total obtainable value from
each of these years. One point to note is that the value is not
particularly stable from year to year.
\begin{table}
\tstyle % For tabular (non-proportional) figures
\centering
\begin{tabular}{l*{2}{S[round-mode = places, round-precision = 1,
    fixed-exponent = 3,
    table-format = 2.1,
    table-omit-exponent,
    table-alignment = right,
    text-rm = \tstyle]}} \toprule
         & \multicolumn{2}{c}{Value (\SI{e6}[\pounds]{\per\year})} \\ \cmidrule{2-3}
 & {Rate-limited} & {Storage-limited} \\ 
 & {(per \si{\GW})}      & {(per \si{\GWh})} \\ \midrule
% 2003    & 19419.881 &                         5438.9906 \\ Full year of price data not available
2004      & 17654.280 &                         6028.7229 \\
2005      & 53317.927 &                         9957.0150 \\
2006      & 58508.910 &                        10179.2800 \\
2007      & 50164.324 &                        10251.3725 \\
2008      & 98665.124 &                        19306.3550 \\
2009      & 45088.914 &                         9696.6875 \\ 
2010      & 27196.531 &                         5828.7850 \\
2011      & 14769.648 &                         4398.7636 \\ \bottomrule
% 10 2012 & 3732.725  &                          879.1075 \\ Full year of price data not available
\end{tabular}
\caption{The maximum value that would have been capturable by \SI{1}{\GW} of
  rate-limited storage (that is, storage having unlimited capacity)
  and \SI{1}{\GWh} of storage-limited storage (that is, storage which
  can charge and discharge effectively instantaneously). Note that
  these are not directly comparable.} 
\label{tab:values}
\end{table}

\subsection{Volume-limited storage}

Volume-limited storage fills up quickly (compared to price
movements). If the operator of such storage bought at the lowest price
of the year and held on to that energy to sell at the highest price,
the storage would be unavailable to take advantage of any other price
differences during the year. So volume-limited storage will more
profitably be used if the operator buys and sells every cycle.

Whereas rate-limited storage will either be buying or selling in
nearly every period, volume-limited storage will only buy at isolated
lows and sell at isolated highs. Compared to rate-limited storage, it
will ``miss out'' because it will not capture intermediate price
differentials; on the other hand, it will always buy and sell as much
as it can at the extremes. 

The above observations can be made more precise. Pretend that
electricity is bought and sold in discrete time steps $t = 1, 2,
3,\dots$, at a price, $p_t$, that is fixed for each time step. Suppose
that a particular volume-limited storage unit has no round-trip
losses. Then it is fairly easy to see that the optimal strategy for
the storage operator (having perfect foresight) is to fill up at each
local mimimum of price and to sell everything at each local
maximum. (I claim that the operator should buy or sell the maximum
possible at each time step.) Now consider the case when $p_{t+1} > p_t$
and suppose that the storage is empty at time~$t$. If the optimal
state at $t+1$ is to be empty, it is better to buy at $t$ and sell at
$t+1$; if the optimal state at $t+1$ is to be full, it is better to
buy at $t$. In any event, therefore, it is better to buy at~$t$. The
same argument applies \emph{mutatis mutandis} to the case when the
storage is full at $t$, and to the two cases when $p_{t+1} < p_t$.)

When the storage is not lossless, the procedure is slightly more
involved.

Let the proportion of energy stored that is available for production
be~$\alpha$ (so that the round-trip loss as a proportion of energy
stored is $1-\alpha$). The optimal strategy will only consider buying
when the price is a local minimum (a strategy which buys when the
price was lower at the previous timestep can be converted to a more
profitable one by buying at the previous timestep; a strategy which
buys when the price is lower at the next timestep can be made more
profitable by buying at the later timestep.)

So consider a local minimum of price, at time $t_1$, say. The idea is
to buy at $t_1$ if it is possible to make a profit by doing so, unless
there is some way to make a greater profit. The procedure is as
follows: Find the first time after $t_1$ at which the price is greater
than $p_{t_1}/\alpha$: say,~$t_2$. (If there is no such time, then one
can only make a loss by buying at $t_1$, so do not buy.) Now buy at
$t_1$ \emph{unless} there is a minimum between $t_1$ and $t_2$ whose
price is lower than that at $t_1$. If this is the case, do not buy at
$t_1$ but set $t_1$ to be the time of the first such minimum and
repeat the procedure.

In any case, next consider the maximum at $t_2$ and repeat the above,
swapping `maximum' and `minimum' as necessary.

Notice that if the ratio between successive maxima and minima is
greater than $1/\alpha$ then the above procedure prescribes buying at
every minimum and selling at every maximum. 

\begin{figure}[htb]
\centering
\includegraphics{minmax.pdf}
\caption{Daily maxima and minima in the market index data
  price. (Labels on the date axis refer to the beginning of the
  calendar year.)}
\label{fig:minmax}
\end{figure}

Figure~\ref{fig:minmax} shows the maximum and minimum daily prices
from 2003 to early~2012. For pumped hydro the round-trip efficiency is
about $\alpha = 0.75$. It turns out that the ratio between the maximum
and minimum price on any given day was less than $1/0.75$ on only 15
days in this period, and 13 of those days were in~2011. Thus a
reasonable procedure for storage-limited storage to follow is simply
to buy at the low of each day and sell at the high, assuming there is
a profit to be made on that day. The results of following this
procedure are also shown in table~\ref{tab:values}.

\subsection{Estimating the short-run value}


\end{document}


